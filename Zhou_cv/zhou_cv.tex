%% start of file `template-zh.tex'.
%% Copyright 2006-2013 Xavier Danaux (xdanaux@gmail.com).
%
% This work may be distributed and/or modified under the
% conditions of the LaTeX Project Public License version 1.3c,
% available at http://www.latex-project.org/lppl/.


\documentclass[11pt,a4paper,sans]{moderncv}   % possible options include font size ('10pt', '11pt' and '12pt'), paper size ('a4paper', 'letterpaper', 'a5paper', 'legalpaper', 'executivepaper' and 'landscape') and font family ('sans' and 'roman')
\usepackage{setspace}
% moderncv 主题
\moderncvstyle{casual}                        % 选项参数是 ‘casual’, ‘classic’, ‘oldstyle’ 和 ’banking’
\moderncvcolor{blue}                          % 选项参数是 ‘blue’ (默认)、‘orange’、‘green’、‘red’、‘purple’ 和 ‘grey’
%\nopagenumbers{}                             % 消除注释以取消自动页码生成功能

% 字符编码
\usepackage[utf8]{inputenc}                   % 替换你正在使用的编码
\usepackage{CJKutf8}

% 调整页面出血
\usepackage[scale=0.75]{geometry}
\setlength{\hintscolumnwidth}{3cm}           % 如果你希望改变日期栏的宽度

% 个人信息
\name{周世奇 }{\LARGE }
%\title{中科院在读博士, 2019.7 毕业\\ \small{浙江绍兴人,1991.12.31, 
%     男}\\
%     \small{电话:15600602175}\\
%     \small{邮箱:shiqizhou@lsec.cc.ac.cn}}                   % 可选项、如不需要可删除
\address{北京市海淀区中关村东路55号}{邮编:100190}            % 可选项、如不需要可删除本行
\phone[mobile]{18518316647}              % 可选项、如不需要可删除本行
%\phone[fixed]{+2~(345)~678~901}               % 可选项、如不需要可删除本行
%\phone[fax]{+3~(456)~789~012}                 % 可选项、如不需要可删除本行
\email{shiqizhou@lsec.cc.ac.cn}                    % 可选项、如不需要可删除本行
%\homepage{www.xialongli.com}                  % 可选项、如不需要可删除本行                % 可选项、如不需要可删除本行
\photo[64pt][0.1pt]{zhou_cv.jpg}                  % ‘64pt’是图片必须压缩至的高度、‘0.4pt‘是图片边框的宽度 (如不需要可调节至0pt)、’picture‘ 是图片文件的名字;可选项、如不需要可删除本行
%\quote{引言(可选项)}                          % 可选项、如不需要可删除本行

% 显示索引号;仅用于在简历中使用了引言
%\makeatletter
%\renewcommand*{\bibliographyitemlabel}{\@biblabel{\arabic{enumiv}}}
%\makeatother

% 分类索引
%\usepackage{multibib}
%\newcites{book,misc}{{Books},{Others}}
%----------------------------------------------------------------------------------
%            内容
%----------------------------------------------------------------------------------
\begin{document}
\begin{CJK}{UTF8}{gbsn}                       % 详情参阅CJK文件包
\maketitle
\begin{spacing}{1.1}
\section{基本信息}
\cvdoubleitem{籍贯:}{浙江绍兴}{国籍:}{中国}
\cvdoubleitem{最高学历:}{博士研究生}{研究方向:}{反问题,成像}
\cvdoubleitem{电话:}{18518316647}{邮箱:}{shiqizhou@lsec.cc.ac.cn}
\cvdoubleitem{学校所在城市:}{北京}{预计毕业时间:}{2019-07-01}
\section{教育背景}
\cvdoubleitem{起止日期:}{2014-09-01-----2019-07-01}{毕业院校:}{中国科学院大学}
\cvdoubleitem{在读院系:}{数学与系统科学研究院}{专业:}{计算数学}
\cvdoubleitem{学历:}{博士研究生}{学位:}{博士 (直博)}
\cvdoubleitem{导师:}{陈志明 院士}{所在实验室:}{科学与工程计算国家重点实验室}
\vspace{1cm}
\cvdoubleitem{起止日期:}{2010-09-01-----2014-07-01}{毕业院校:}{电子科技大学(985)}
\cvdoubleitem{就读院系:}{数学科学学院}{专业:}{数学与应用数学}
\cvdoubleitem{学历:}{本科}{成绩排名:}{1\%}
\cvitem{学位:}{双学士(理学和经济学)}
\section{个人简介及专业技能}
\cvitem{数学}{扎实的数学基础及严谨的逻辑分析,推导能力;优秀的数学建模能力及算法设计能力;掌握数学分析、线性代数、微分方程、数理统计、数值计算、渐近分析等数学理论}
\cvitem{编程开发}{熟悉Python/Matlab/C等编程语言,熟悉Linux开发环境}
\cvitem{外语能力}{英语 CET 6; 良好的英语文献读写能力,及学术报告}
\cvitem{跨学科}{熟悉基本的机器学习,深度学习算法及相关算法实现}
\cvitem{其他}{具备较强的学习能力,良好的交流沟通能力及团队合作意识}
\section{博士期间项目经历}
\cvitem{起止日期:}{2014-09-01-----2017.9.1}
\cvitem{项目名称:}{国家基础研究(973)项目:适应于千万亿次科学计算等新型计算模式(No.CB2011CB309700)半空间弹性波的逆时偏移成像算法研究}
\cvitem{项目介绍:}{\small 本项目主要研究如何快速探寻嵌入在半空间弹性介质中的障碍物, 包括其位置、形状。本项目所用数学手段为反散射理论, 其应用背景为石油勘探、无损探伤检测以及医疗成像。主要研究问题是: 地底矿产勘探(均匀半空间模型)。我们利用逆时偏移(RTM)的思想提出稳定的成像函数并给出分辨率分析及相关数值算例子;该成像函数不依赖障碍物的边界条件。}
\cvitem{主要贡献:}{\small 给出了半空间中弹性波方程适定性(唯一性,存在性,连续性)的完整证明; 提出了弹性波的点扩散函数(Point Spread Function)及对其函数性态的严格证明;给出了成像函数的分辨率分析;实现了相关数值算例来说明成像函数的效果。}
\vspace{1cm}
\cvitem{起止日期:}{2016-05-01-----至今}
\cvitem{项目名称:}{国家自然科学基金群体创新基金:科学工程计算的方法与应用(No.11321061)无相位弹性波数据成像}
\cvitem{项目介绍:}{\small 在现实的地球勘探中,往往得到的都是不带相位的地震波强度(在数学表达中就是向量的模)。研究仅利用无相位数据的情况下, 提出一个稳定的成像效果。 对所提出的成像函数进行严格的数学理论分析及数值验证。}
\cvitem{主要贡献:}{\small 提出了无相位数据的成像函数 证明了近似Helmholtz-Kirchhoff恒等式;实现了相关数值算例来说明成像函数的效果。}
\section{Publication}
\cvitem{论文名称:}{Reverse Time Migration for Extended Obstacles in
	the Half Space: Elastic Waves, with Z. Chen, submitted, 2017.}
\cvitem{}{A Direct Imaging Method for Elastic Scattering Data without Phaseless Data, with Z. Chen, submitted, 2018.}
\section{获奖情况}

\cvdoubleitem{省级:}{人民特等奖学金 }{时间}{2011.11}
\cvdoubleitem{国家级:}{国家奖学金 }{时间}{2012.11}
\cvdoubleitem{省级:}{第四届全国大学生数学竞赛四川赛区   一等奖 }{时间}{2012.12}
\cvdoubleitem{国家级:}{第四届全国大学生数学竞赛决赛 三等奖 }{时间}{2013.03}
\cvdoubleitem{国家级:}{韩国三星奖学金 }{时间}{2013.11}
\cvdoubleitem{国家级:}{第五届全国大学生数学竞赛决赛 二等奖 }{时间}{2014.03}
\cvdoubleitem{校级:}{电子科技大学优秀毕业生 }{时间}{2014.07}


%\nocite{*}
%\bibliographystyle{plain}
%\bibliography{publications}                    % 'publications' 是BibTeX文件的文件名

% 来自BibTeX文件并使用multibib包的出版物
%\section{出版物}
%\nocitebook{book1,book2}
%\bibliographystylebook{plain}
%\bibliographybook{publications}               % 'publications' 是BibTeX文件的文件名
%\nocitemisc{misc1,misc2,misc3}
%\bibliographystylemisc{plain}
%\bibliographymisc{publications}               % 'publications' 是BibTeX文件的文件名
\end{spacing}
\clearpage\end{CJK}

\end{document}


%% 文件结尾 `template-zh.tex'.
